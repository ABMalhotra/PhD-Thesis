\clearpage
\thispagestyle{empty}
\begin{centering}
\textbf{SUMMARY}\\
\vspace{\baselineskip}
\end{centering}

Advances in nanotechnology have opened doors for a rational design of semiconductor materials with engineered thermal properties. This ability to control nanoscale thermal transport properties is key for advancing multiple technologies including thermoelectrics which have immense potential as power generators as well as for waste-heat recovery, optoelectronics such as quantum cascade lasers which are used in applications from long-range communication to medical surgery, and the ubiquitous microelectronic transistors which power our laptops and phones. Thus, a fundamental understanding of the conduction of heat in semiconductor nanostructures is foundational for accelerating human progress.

The conduction of heat in solid state semiconductors is controlled by the thermal vibrations that originate on the application of a thermal gradient. These thermal vibrations, or phonons, are the carriers of heat whose transport properties can be controlled via their interactions with surfaces in the nanostructure. Thus, the flow of heat can be modulated by designing the nanostructure and engineering the phonon-surface interaction. Predicting the flow of heat at nanometer length scales by modeling the interaction of phonons with surfaces in a nanostructure in order to understand thermal conduction in semiconductor nanostructures is the central objective of this thesis. 

To achieve this objective, we first create a predictive model for nanoscale thermal transport for nanowires and thin-films in \Cref{chap:predictive}. The created model incorporates experimentally quantifiable descriptors, such as surface roughness and correlation length, in order to evaluate the role of phonon reflection from surfaces on thermal transport. We extend the predictive model in \Cref{chap:diff_boundary} to a  generic case where both the surfaces of the thin-film are distinct, a situation that arises in experimental growth of free-standing thin-films, and show the  spatial evolution of thermal flux in thin-films as a function of surface conditions analogous to fluid flow in confined geometries. We use the case of distinct boundaries to model nanotubes in \Cref{chap:nt}, which have recently gained experimental attention. Importantly, in all of these cases we can predict not only the thermal conductivity but also the proportion of heat carried by individual phonons which is key for rational thermal material design.

In the second half of the thesis we focus on planar layered nanostructures to understand the effects of phonon transmission (in addition to reflection) on thermal properties. In \Cref{chap:layered} we model planar bi-layer and tri-layer nanostructures and show that contrary to the commonly held notion that thermal conductivity at the nanoscale can only be reduced, these layered structures present an opportunity to locally enhance thermal conductivity. We uncover the phenomenon of phonon injection that is central to these effects. We also apply the newly developed understanding to film-on-substrate systems, which are a common semiconductor growth platform and highlight the importance of phonon coupling effects on thermal properties of this system. In \Cref{chap:slxp}, we study cross-plane thermal transport in superlattices and predict the role of all relevant length scales, i.e. surface roughness, period length and superlattice size on thermal properties, which would be of profound practical importance.

Overall, we have created thermal transport models for a variety of nanostructures which include cylindrical cross-section structures -- nanowires and nanotubes, and planar structures -- thin-films, bi-layers, tri-layers and superlattices. We elucidated the role of phonon scattering at surfaces (both reflection and transmission) in experimentally realizable nanostructures using quantifiable surface descriptors. The methodology and the findings of this thesis advance nanoscale thermal transport literature by providing a fundamental and quantitative understanding about the transport of heat in semiconductor nanostructures.


%\todo{revisit.}
\pagenumbering{gobble}  %remove page number on summary page