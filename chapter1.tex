%!TEX root = thesis.tex
\chapter{Introduction}
\label{chap:intro}

\section{Heat and Nanotechnology}

The understanding and control of thermal transport processes has been central to human progress over the course of our evolution. From the discovery of fire by friction, to harnessing the thermal energy via the steam engine, the path of human progress has been paved with thermal energy. Now, with the burgeoning advances in nanotechnology, understanding and controlling heat flow at the nanoscale has become a key scientific and technological challenge. Just as the in-depth understanding of transport of electrons and photons revolutionized the fields of electronics and photonics \cite{RN486,book_maldovan}, similar degree of progress in nanoscale phononics (phonon is the thermal carrier in semiconductor solids) has the potential to propel energy research and create fundamental and novel discoveries about flow of heat. Technologically, nanoscale thermal conduction is central to the performance of microelectronic processors, optoelectronic devices such as lasers and light emitting diodes, and for energy materials such as thermoelectrics. Take for instance the modern-day nanometer scale computer processors, these high performance devices can have localized spatial locations called hot-spots where the magnitude of thermal flux \gls{J} can exceed 10 \si{\watt\per\centi\meter\squared} \cite{rev_Majum_thermoelec,rev_Moore} a value greater than the Solar flux reaching the Earth (Solar constant = 0.13 \si{\watt\per\centi\meter\squared}) \cite{solarconstant}. The efficient dissipation of heat is central to the usability of these devices. In addition to the lack of space to incorporate heat dissipation solutions, the problem is further compounded by the reduction of thermal conductivity at these nano length scales as compared to the bulk \cite{book_Zhuomin}. This reduction of thermal conduction arises from the interaction of heat carriers (i.e. phonons) with nanostructures, which inhibits their heat transporting ability. Furthermore, heat dissipation also impacts the efficiency of optoelectronic devices such as photo-detectors and light emitting diodes \cite{book_rogalski_infrared}. The reduced nanoscale thermal conductivities are not always negative in their impact, as low thermal conductivity coupled with high electrical conductivity allows thermoelectric devices to convert waste-heat into electricity efficiently. These thermoelectrics further have the capability to generate power for off-grid applications and to even cool down microelectronic hot-spots \cite{rev_snyder,rev_thermoelec_tritt,rev_Majum_thermoelec}. Due to its ubiquity, thermal energy transport thus remains central to current technological development and these examples are illustrative of the potential impact that a rational thermal design of nanoscale materials can have in advancing important technologies central to the modern civilization.

  \section{Nanoscale Thermal Conduction and Phonons}
Thermal conduction is the dominant mechanism of heat conduction in solids, and occurs when a temperature gradient, which acts as the thermal driving force, is applied across the solid. Fourier's Law is used in bulk systems to describe the heat conduction \cite{book_Bird},

\begin{equation}
J_x = -\kappa \nabla_x T
\label{eq:fourier}
\end{equation} 

where \gls{Jx} is the the conductive thermal flux generated by the applied temperature gradient \gls{gradT} in the $x$-direction, and \gls{K} is the corresponding thermal conductivity in the same direction.  The microscopic origins of thermal conduction lie in the vibrations of atoms that create a net transport of energy from the hotter end to the cooler end \cite{book_Reif}. From this atomic perspective, thermal transport in non-metallic crystalline solids is described as the transport of atomic vibrations with broadband frequencies that are generated and move under an applied temperature gradient. Quantized lattice vibrations are called \textit{phonons} and can be considered to be the particles of heat \cite{book_Ashcroft}. Thermal conduction in semiconductor nanostructures can thus be described as the net transport of phonons from the high temperature region to the low temperature region. 
\par Phonons are characterized by their frequency \gls{freq} of vibration, their propagation wavevector \gls{k}, and their longitudinal or transversal polarization. Phonons in solid materials can exist in a broadband range of frequencies \gls{freq} determined by the dispersion relation $\omega(\mathbf{k})$, a fundamental material property giving the relation between the phonon frequency  \gls{freq} and the phonon wavevector \gls{k} for each polarization. Being able to establish how much heat can be conducted in a solid material requires knowledge of three fundamental phonon transport properties -- how much energy phonons can carry (\gls{hbar}\gls{freq}$_\mathbf{k}$), how fast they can move (group velocity \gls{vg}$_\mathbf{k}$), and how far they can travel (mean-free-path \gls{mfp}$_\mathbf{k}$). We note that in nanostructures, length scales smaller than mean-free-paths of bulk phonons create interactions between phonons and surfaces, which are central to the modification of thermal properties at the nanoscale. We further note that all the phonon transport properties are dependent on wavevector \gls{k} and when combined together determine ability of the material to conduct heat. Thus, the transport of heat under a thermal gradient is an aggregate behavior of phonons generated within the nanostructure at each spatial location $r$, moving along all directions with their wavevector dependent thermal transport properties, \gls{hbar}\gls{freq}$_\mathbf{k}$, \gls{vg}$_\mathbf{k}$, and \gls{mfp}$_\mathbf{k}$, and their statistically averaged transport is contained in the thermal conductivity \gls{K},

\begin{equation}
{J_x}=-\kappa \nabla_x T = -\frac{1}{t}\,{\int_{0}^{t} \sum_{pol} \Big[ \frac{1}{8 \pi^3}\iiint \hbar\omega_{\mathbf{k}}\dfrac{\partial f^{BE}}{\partial T}v^x_{\mathbf{k}}\ell^x_{\mathbf{k}}({r}) \,d^3{k} \Big]} \,d{r}\:\nabla_x T
\label{eq:phonon_fourier}
\end{equation} 

where, the contribution of each phonon at each spatial location $r$ is integrated over the complete \gls{k}-space, summed for each phonon polarization $pol$, and averaged over the size \gls{t} of the nanostructure to determine the effective thermal conductivity \gls{K} of the nanostructure in the direction of the gradient.\footnote{For calculations of thermal transport along a particular direction, say $x$, the components of group velocity \gls{vg}$^x_\mathbf{k}$ and mean-free-path \gls{mfp}$^x_\mathbf{k}$ are used.} Note that the equilibrium Bose-Einstein population \gls{fBE} is a function of temperature and phonon energy. Importantly, \Cref{eq:phonon_fourier} highlights a key fundamental physical aspect, that is nanoscale thermal conduction is a spatially local phenomenon. Under such spatial locality, phonons at different spatial location in general, have distinct mean-free-paths, and therefore contribute distinctly to thermal transport \cite{maldovan2011tf,ownSpatialTF}. This physical nature arises due to the differing influence of surfaces on different spatial  locations, and distinguishes nanoscale thermal transport from bulk-scale transport. This distinction also underlies the observation that thermal conductivity at the nanoscale is a strong function of nanostructure size \cite{book_Ziman,ownNW,ownTF}. The structure-size dependent thermal conductivity is characteristic of the \emph{quasi-ballistic transport} regime, which is the primary focus of this thesis. Contrarily, in the \emph{diffusive transport} regime, the thermal conductivity is not linked to structure size as seen at the bulk-scale \cite{book_Bird}.
 
\section{The Phonon Boltzmann Transport Equation}
A quasi-ballistic regime of phonon transport is characterized by the existence of two kinds of phonon scattering events. The first is the scattering of phonons due to nanostructure surfaces \cite{book_Ziman}. This phonon-surface interaction is spatially discrete in nature, since it can occur only at specific locations in the nanostructure, i.e. at boundaries. The second kind of interaction that occurs throughout the volume of the nanostructure is the anharmonic phonon-phonon scattering \cite{book_Ashcroft}. Other volumetric events include phonon-alloy and phonon-impurity scattering. Both kinds of scattering events -- discrete and volumetric -- influence the thermal transport in nanostructures by modifying the transport of phonons.  Thus, in order to model thermal transport for a wide range of conditions, the modeling tool of choice should at minimum possess the capability to include the scattering physics, with the flexibility of adding new features as required.

To this end, the Boltzmann transport equation (BTE) provides a rigorous framework to mathematically represent the transport of phonons. At steady state and under a single-mode relaxation time approximation, the BTE for phonons can be written as \cite{book_Ziman,ownCoupling1},

\begin{equation}
\mathbf{v}(\mathbf{k}) \cdot \nabla f(\mathbf{k},\mathbf{r})  + \frac{f(\mathbf{k},\mathbf{r})-f^{BE}(\mathbf{k},\mathbf{r})}{\tau(\mathbf{k})}=0
\label{eq:bte_ss}
\end{equation} 				
where \gls{f} is the phonon population function deviated from the equilibrium Bose-Einstein population \gls{fBE} due to the applied driving force. The relaxation time \gls{tau} accounts for the anharmonic phonon scattering effects, and can be obtained from first-principle calculations \cite{stokes_bulkSi_tau,bAs_broido}, or iteratively by matching experimental bulk conductivity measurements to model predictions \cite{aksamijaNW,maldovan2011tf}.\footnote{While both these approaches have been shown to yield similar outputs for Silicon (see \Cref{app:si_relaxation_rates}) first principle approach can be useful for new materials or materials without sufficient experiments, while the iterative approach is useful to  obtain temperature dependent functional forms of relaxation time cheaply.} For emphasis, the dependence of quantities on wavevector \gls{k} and position $r$ are explicitly written. A closed form solution to \Cref{eq:bte_ss} is nanostructure dependent and spatial boundary conditions for phononic populations are required \cite{ownKK1,ownCoupling1,ownCoupling2,RN396} . Therefore, we discuss the exact methodology for obtaining non-equilibrium phononic populations \gls{f} in specific chapters as required. Once the non-equilibrium populations are calculated as a function of position for each phonon mode, thermal conductivity can be obtained using,

\begin{equation}
J_x=-\kappa \nabla_x T = -\frac{1}{t}\,{\int_{0}^{t} \sum_{pol} \Big[ \frac{1}{8 \pi^3}\iiint \hbar\omega_{\mathbf{k}}v^x_{\mathbf{k}}f_{\mathbf{k}}({r}) \,d^3{k} \Big]} \,d{r}
\label{eq:pop_fourier}
\end{equation} 

	A comparison between \Cref{eq:phonon_fourier} and \eqref{eq:pop_fourier} shows that there is a fundamental connection between mean-free-path \gls{mfp} and phononic population \gls{f}. We explicitly show in thin-films (see \Cref{chap:diff_boundary}) that both equations are equivalent and either of them can be used to obtain thermal properties. Thus, for nanowires and nanotubes (see \Cref{chap:predictive,chap:nt}), \Cref{eq:phonon_fourier} is evaluated since geometrically obtaining mean-free-paths is simpler. On the other hand, in layered nanostructures (see \Cref{chap:layered,chap:slxp}) the calculation of phononic population \gls{f} by solving the BTE allows for a more tractable solution allowing us to evaluate \Cref{eq:pop_fourier} to obtain thermal properties in these nanostructures.
	
\section{Outline of the Thesis}
The overarching vision of this thesis is to explore the phenomenon of nanoscale thermal conduction in different nanostructures. Aligned with this vision, the work contained in this dissertation is divided into two main parts,
\begin{itemize}
\item First, we explore nanostructures in which the interaction of phonons with surfaces is focused on reflection i.e. backward scattering.  The nanostructures of this category included in this thesis are nanowires, thin-films and nanotubes. 
	\begin{itemize}
	
	\item In \Cref{chap:predictive}, we formulate a predictive mathematical model that incorporates experimentally quantifiable surface descriptors as inputs for phonon-surface scattering. Such a model allows for quantifying the role of surfaces in nanoscale thermal transport while allowing for a comparison with measurements.
	
	\item In \Cref{chap:diff_boundary}, we extend the model to a general case of thin-films with distinct surfaces. We use the model to show the spatial distribution of thermal flux and draw the analogy with fluid flow.
	
	\item In \Cref{chap:nt}, we utilize the developed model to evaluate thermal conduction in semiconductor nanotubes. This study allows for an understanding of cylindrical geometries with distinct inner and outer surfaces and is utilized in explaining experimental measurements.
	\end{itemize}
	
	\item Second, we explore nanostructures in which phonons upon interacting with surfaces can be reflected and transmitted, i.e. both backward and forward scattered. The nanostructures of this category included in this thesis are semiconductor bi-layers, tri-layers and superlattices.
	\begin{itemize}
	\item In \Cref{chap:layered}, we model layered nanostructures under an in-plane temperature gradient to understand the role of phonon transmission. This novel study allows us to uncover a novel phenomenon of phonon coupling, which can be utilized to locally modulate thermal properties at the nanoscale.
	\item In \Cref{chap:slxp}, we model cross-plane thermal transport in superlattices allowing for an exploration across multiple length scales -- roughness, period and size. 
	\end{itemize}
	
\end{itemize}





